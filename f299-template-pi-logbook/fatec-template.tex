%%%% fatec-article.tex, 2024/03/10

%% Classe de documento
\documentclass[
landscape,
  a4paper,%% Tamanho de papel: a4paper, letterpaper (^), etc.
  12pt,%% Tamanho de fonte: 10pt (^), 11pt, 12pt, etc.
  english,%% Idioma secundário (penúltimo) (>)
  brazilian,%% Idioma primário (último) (>)
]{article}

%% Pacotes utilizados
\usepackage[]{fatec-article}
\usepackage{setspace}
\usepackage{graphicx}  % Necessário para usar \resizebox

%% Processamento de entradas (itens) do índice remissivo (makeindex)
%\makeindex%

%% Arquivo(s) de referências
%\addbibresource{fatec-article.bib}

%% Início do documento
\begin{document}

% Seções e subseções
%\section{Título de Seção Primária}%

%\subsection{Título de Seção Secundária}%

%\subsubsection{Título de Seção Terciária}%

%\paragraph{Título de seção quaternária}%

%\subparagraph{Título de seção quinária}%

%\section*{Diário de Bordo}%

% Tabela com ajuste de tamanho
\begin{table}[]
\centering
\resizebox{\textwidth}{!}{ % Ajusta a tabela para a largura da página
\begin{tabular}{|l|l|l|l|l|}
\hline
Nome da Atividade                   & Data de início & Data de término & Responsável pela atividade & Descrição da atividade realizada \\ \hline
Planejamento no Notion              & 16/08 & 16/08 & Ricardo e Lucas & Definição das funções de cada membro e cronograma no Notion \\ \hline
Pesquisa inicial                    & 17/08 & 21/08 & Todos & Separação de 20 artigos 4 por cada membro da equipe; leitura de todos os artigos \\ \hline
Começo do artigo científico         & 25/08 & 01/09 & Ricardo Estevam e Lucas Frazão & Conclusão da primeira versão da introdução \\ \hline
Aprimoramento da introdução         & 02/09 & 05/09 & Ricardo Estevam e Lucas Frazão & Termino da segunda versão da introdução \\ \hline
Aprimoramento Final da introdução   & 06/09 & 09/09 & Ricardo Estevam e Lucas Frazão & Conclusão da introdução e revisão \\ \hline
Definindo os objetivos              & 11/09 & 12/09 & Ricardo Estevam e Lucas Frazão & Inicio e conclusão dos objetivos \\ \hline
Fluxograma de metodologia           & 13/09 & 13/09 & Todos & Juntamos o grupo e fizemos o rascunho da metodologia \\ \hline
Metodologia no Figma                & 14/09 & 14/09 & Ricardo Estevam & Foi repassado a ideia do papel para o Figma \\ \hline
Metodologia no Artigo               & 16/09 & 17/09 & Ricardo Estevam e Lucas Frazão & Escrevendo passo a passo da metodologia do projeto \\ \hline
Ideação do estado da arte           & 20/09 & 20/09 & Ricardo Estevam e Lucas Frazão & Foi feito a ideia de como escreveríamos o estado da arte \\ \hline
Site Figma                          & 20/09 & 27/09 & Bruno de Lucca & Foi terminado como seria o site web no Figma \\ \hline
Separação dos artigos               & 21/09 & 22/09 & Ricardo Estevam e Lucas Frazão & Foi feita a separação dos melhores artigos e a leitura deles \\ \hline
Inicio do estado da arte            & 23/09 & 25/09 & Ricardo Estevam e Lucas Frazão & Conclusão da primeira versão do estado da arte \\ \hline
Modificação estado da arte          & 24/09 & 25/09 & Ricardo Estevam e Lucas Frazão & Foi adicionado alguns artigos a mais e refinamos na escrita \\ \hline
Finalização estado da arte          & 26/09 & 01/10 & Ricardo Estevam e Lucas Frazão & Alteramos alguns artigos da versão anterior e substituímos a tecnologia Machine-Learning por Deep-Learning \\ \hline
Paleta de cores                     & 02/10 & 05/10 & Todos & Nesse dia sugerimos várias paletas de cores para o grupo e para o projeto \\ \hline
Protótipo de baixa fidelidade       & 03/10 & 04/10 & Lucas Frazão & Rascunho do projeto no Figma \\ \hline
Tela login e cadastro               & 05/10 & 05/10 & Ricardo Estevam e Lucas Frazão & Fizemos a primeira tela de login do Figma; nesse dia foi dado início à construção do site web \\ \hline
Tela escaneamento                   & 06/10 & 07/10 & Ricardo Estevam e Lucas Frazão & Fizemos animações na tela de escaneamento \\ \hline
Tela login APEX                     & 07/10 & 10/10 & Leo Lima e Ana Flávia & Conclusão da tela de diagnóstico, declaração do nome AVASS para o aplicativo. Desenvolvimento da tela de login no APEX \\ \hline
Mudança em login APEX               & 11/10 & 15/10 & Leo Lima e Ana Flávia & Desenvolvimento da tela de cadastro e criação do banco de dados da tela cadastro \\ \hline
Animação splash screen              & 13/10 & 13/10 & Ricardo Estevam & Foi criada animações antes do usuário acessar a tela principal do aplicativo no Figma \\ \hline
Mapa de calor Figma                 & 14/10 & 14/10 & Ricardo Estevam & Foi criado um mapa de calor utilizando frame e um mapa real no Figma \\ \hline
Tela diagnóstico Figma              & 15/10 & 16/10 & Ricardo Estevam e Lucas Frazão & Criação de diagnóstico, foi feito animações para deixar um efeito de funcionalidade em tempo real \\ \hline
Interações Figma                    & 16/10 & 16/10 & Ricardo Estevam e Lucas Frazão & Montado as interações das telas do Figma; tornamos elas navegáveis \\ \hline
Barra de Navegação                  & 17/10 & 19/10 & Ricardo Estevam & Barra de navegação para o usuário, onde ele pode acessar a "home", "arquivos" e "perfil" (obs: Próximo do dia 20 foi mudada a barra de navegação, possuindo as mesmas funcionalidades, foi adicionada animações e também foi ajustado o tamanho) \\ \hline
Customização Login APEX             & 20/10 & 20/10 & Leo Lima e Ana Flávia & Mudança nas cores de fundo da tela do APEX \\ \hline
Ajustes mapa de calor Figma         & 20/10 & 20/10 & Ricardo Estevam & Ajuste no mapa de calor \\ \hline
Tela Suporte Figma                  & 21/10 & 22/10 & Lucas Frazão & Conclusão da tela de suporte \\ \hline
Artigo em LaTeX                     & 23/10 & 17/11 & Ricardo Estevam & Passando o Artigo para LaTeX \\ \hline
Referenciando artigo                & 24/10 & 27/10 & Ricardo Estevam & Término de referências \\ \hline
Início modelagem conceitual e lógica& 24/10 & 30/10 & Lucas Frazão e Ana Flávia & Ajuste nas cardinalidades \\ \hline
Validação do banco de dados APEX    & 25/10 & 27/10 & Leo Lima e Ana Flávia & Revisão de todos os processos \\ \hline
Início pitch                        & 31/10 & 17/11 & Bruno de Lucca & Ideação pitch \\ \hline
Banner Apresentação                 & 01/11 & 14/11 & Lucas Frazão e Ana Flávia & Criação banner Figma \\ \hline
Áudio pitch                         & 03/11 & 05/11 & Bruno de Lucca & Conclusão da voz com IA \\ \hline
Novo APEX                           & 12/11 & 13/11 & Leo Lima e Ricardo Estevam & Foi refeita a tela de cadastro e de login, por conta da anterior estar desalinhada \\ \hline
Conclusão site Web                  & 14/11 & 14/11 & Ricardo Estevam & Menus navegáveis em JS \\ \hline
Ajustes telas Figma                 & 16/11 & 17/11 & Ricardo Estevam & Ajuste tela mapa\\ \hline
Ajustes no Pitch                    & 17/11 & 17/11 & Ricardo Estevam & Animações utilizando IA, áudio utilizando IA e também foi adicionado legendas com um IA\\ \hline
\end{tabular}
}
\end{table}

\end{document}

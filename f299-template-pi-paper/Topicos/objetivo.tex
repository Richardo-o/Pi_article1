Este estudo visa desenvolver um sistema baseado em inteligência artificial para a detecção e mapeamento da infecção por Xanthomonas phaseoli pv. manihotis em folhas de mandioca. Utilizando a tecnologia Deep Learning, o sistema irá analisar imagens das folhas para identificar e classificar o grau de infecção com base em probabilidades. Além disso, pretende-se integrar os dados de infecção com informações, a fim de mapear a distribuição da praga na região do Vale do Ribeira.
Objetivo Específico:
Mapear a doença, realizando levantamentos da incidência da bacteriose Xanthomonas phaseoli pv. manihotis nas plantações de mandioca no Vale do Ribeira, identificando áreas de maior impacto.
Desenvolver um aplicativo baseado em Inteligência Artificial que utilize Deep Learning para identificar e prever a ocorrência da bacteriose em plantações de mandioca, possibilitando maior agilidade no reconhecimento da bacteriose em grandes áreas de plantações.




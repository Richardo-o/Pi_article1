A ONU (Organizações das Nações Unidas) tem como proposta os objetivos de desenvolvimento sustentável (ODS) e um dos seus objetivos é a “Vida Sobre a Terra”, com a expectativa de: proteger, recuperar e promover o uso sustentável dos ecossistemas terrestres, gerir de forma sustentável as florestas, combater a desertificação, deter e reverter a degradação da terra e deter a perda.  O Brasil tem grande destaque como um dos principais países exportadores agrícolas do planeta.

Em 2023, o país teve uma alta de 4,8\% em relação ao ano anterior, e, somadas, as exportações chegaram ao valor de US\$166,55 bilhões, segundo dados do Ministério da Agricultura e Pecuária \textcite{MAPA:11}. No Brasil, a mandioca (Manihot esculenta Crantz) é uma das plantações mais significativas em termos de volume de produção, ficando atrás apenas da cana-de-açúcar \cite{FURLANETO}. 

Pesquisas realizadas pela CONAB mostram que o Brasil é o quarto maior exportador de mandioca na américa latina. O Estado de São Paulo possui uma área plantada de 51 mil hectares, posicionando-se como o sexto maior produtor de mandioca no país, (INSTITUTO DE ECONOMIA AGRÍCOLA),\textcite{CONAB}. Na região do Vale do Ribeira encontra-se um número significativo na produção de Mandioca onde é cultivada e comercializada internamente por povos Quilombolas. Os quilombolas se localizam em áreas remotas ao longo da Bacia do Rio Ribeira de Iguape (sul do Brasil), cobertas pela vegetação da Mata Atlântica, um dos hotspots de biodiversidade do mundo, desde os primórdios da ocupação (século XVIII), os quilombolas têm sido historicamente dependentes do cultivo de arroz, milho, mandioca e feijão \cite{Prado}.

Esta importante cultura está ameaçada por diversas pragas que impactam a produtividade, e entre esses problemas, as bacterioses são uma das principais preocupações. As bacterioses (doenças originadas por bactérias), recebem uma posição de destaque dentre essas pragas e podem ser nocivas à plantações, podendo acarretar até o desfalecimento das plantas se não forem previamente identificadas e tratadas \textcite{Embrapa}. Dentre os patógenos bacterianos, o Xanthomonas phaseoli pv. manihotis (Xpm) atraí atenção significativa da comunidade científica devido ao seu comportamento vascular e sistêmico, que lhe confere um alto potencial de causar danos significativos em plantações de mandioca, especialmente em regiões tropicais, atraindo assim a relevância científica em todo mundo \cite{MANSFIELD}.

A bacteriose Xanthomonas afeta cerca de 30\% de toda produção de mandioca \textcite{Germoplasma}. O prejuízo na produção das raízes pode afetar em até 70\% do seu peso, tornando a produção inviável quando não se tem o controle dessa doença \textcite{Embrapa}. Um dos problemas enfrentados é a falta de informações da incidência dessas bacteriose, embora mais patógenos bacterianos infectam a mandioca, as informações na literatura são escassas (Cassava diseases caused by Xanthomonas phaseoli pv. manihotis and Xanthomonas cassavae., 2021)\textcite{Cassava1999}. Com base nas pesquisas apontadas anteriormente é notório a preocupação com relação às bacterioses nas plantações e seus impactos no mundo agrícola. 

Com o objetivo de identificar a bacteriose e realizar seu mapeamento, considerou-se a criação de um aplicativo suportado por uma Inteligência Artificial (IA) por meio de Redes Neurais Artificiais. A Deep Learning é uma subcategoria da inteligência artificial que envolve o treinamento de redes neurais profundas para reconhecer padrões complexos e aprender continuamente a partir de grandes volumes de dados. Após o processo de treinamento, essas redes são capazes de executar tarefas semelhantes com as de seres humanos, sendo uma delas analisar imagens com alta precisão. Este aplicativo visa monitorar e, trazer, em probabilidade, o grau da bacteriose Xanthomonas na planta, visto que a bacteriose da mandioca, causada por Xanthomonas phaseoli pv. manihotis, é uma das doenças mais importantes que afetam a produção de mandioca no mundo.(Scielo.,Bactérias endofíticas Klebsiella controlam a bacteriose da mandioca na Amazônia Oriental., 2024)\cite{Klebsiella}.




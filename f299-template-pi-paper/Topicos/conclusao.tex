Com base nos artigos vistos anteriormente, sabemos que é possível criar um aplicativo que consiga identificar uma bacteriose por meio de uma IA e trazer probabilidades precisas. Quando se trata de analisar imagens e suas complexidades, a Inteligência Artificial se mostra bastante eficaz na geração de informações que podem auxiliar um profissional na avaliação das condições das plantações. Mapas de calor também são possíveis, visto que eles oferecem uma representação visual clara das áreas mais afetadas, facilitando a tomada de decisões estratégicas e otimizando o manejo das culturas. Assim, a integração dessas tecnologias não apenas potencializa a eficiência na identificação de problemas, mas também contribui para a sustentabilidade e produtividade agrícola a longo prazo.

O uso de Inteligência Artificial para resolução de problemas complexos vem ganhando força no mundo inteiro. Na agricultura, um sistema que utiliza visão computacional para analisar uma plantação e saber o seu estado requer altos investimentos. Entretanto, o uso de redes neurais pode resultar em custos reduzidos, especialmente quando se considera que apenas um dispositivo móvel, como um smartphone, é suficiente para a captura e análise de imagens.

No artigo Utilização de IA para Controle de Pragas na Agricultura (\textcite{CPA2022}), observa-se o uso de Machine Learning. Esse projeto tem como objetivo utilizar soluções de Machine Learning para coletar e analisar imagens que mostram as condições das folhas da soja e identificar se há ou não alguma praga com base na imagem.

Um método semelhante a esse é o EfficientNet, que é uma subcategoria do Machine Learning. O EfficientNet é uma arquitetura específica de Redes Neurais Convolucionais (CNN) desenvolvida para otimizar o desempenho em tarefas de visão computacional. O projeto Método de detecção de doenças da mandioca baseado no EfficientNet utiliza esse método e se aprofunda no uso de métodos de aprendizagem profunda . O foco principal do projeto mencionado é categorizar doenças foliares usando imagens do conjunto de dados Kaggle \cite{Leaf}.

De forma semelhante, o projeto Classificação de doenças foliares de mandioca, orientado por Aprendizagem Profunda \textcite{EfficientNet}, propõe usar o espaço de cores HSV além do EfficientNet para realizar a tarefa de identificação da praga na plantação de mandioca. Converter imagens para o espaço HSV como parte do pré-processamento pode simplificar o treinamento de modelos de machine learning e EfficientNet, especialmente quando o modelo precisa aprender características específicas da cor.

O Machine Learning mostrado anteriormente nos artigos demonstra ser uma boa opção quando se trata de avaliar imagens; porém, a Deep Learning possui uma maior capacidade de capturar padrões complexos. Sendo assim, o projeto Classificação de Bacteriose nas Folhas da Mandioca visa utilizar a Deep Learning para identificar a bacteriose Xanthomonas phaseoli. Como existem diversas bacterioses semelhantes a Xanthomonas phaseoli, a estrutura fornecida pela Machine Learning pode não ser suficiente para diferenciar essas formas da bacteriose, resultando em uma menor precisão na identificação.